% Created 2025-02-19 Wed 10:35
% Intended LaTeX compiler: pdflatex
\documentclass[11pt]{article}
\usepackage[utf8]{inputenc}
\usepackage[T1]{fontenc}
\usepackage{graphicx}
\usepackage{longtable}
\usepackage{wrapfig}
\usepackage{rotating}
\usepackage[normalem]{ulem}
\usepackage{amsmath}
\usepackage{amssymb}
\usepackage{capt-of}
\usepackage{hyperref}
\usepackage{amsmath}
\usepackage{xcolor}
\usepackage{fvextra}
\usepackage{minted}
\usepackage{concrete}
\usepackage[T1]{fontenc}
\setminted{frame=single}
\newcommand{\aura}[0]{\textsf{AURA}}
\newcommand{\grandpa}[0]{\textsf{GRANDPA}}
\newcommand{\docversion}{1.2}
\usepackage{datetime2}
\renewcommand{\today}{\DTMtoday}
\usepackage{fancyhdr}
\fancypagestyle{customfooter}{
\fancyhf{}
\renewcommand{\headrulewidth}{0pt}
\fancyhead[C]{\hfill Version \docversion{} | \today}
\fancyfoot[C]{Copyright © 2025 Input Output Global | Confidential \hfill \thepage}
}
\pagestyle{customfooter}
\usepackage{caption}
\captionsetup[figure]{aboveskip=10pt, belowskip=15pt}
\captionsetup{labelfont=normalfont, font=it}
\author{Jon Rossie}
\date{v.\docversion{}, \today}
\title{Committee Selection Risks in Midnight}
\hypersetup{
 pdfauthor={Jon Rossie},
 pdftitle={Committee Selection Risks in Midnight},
 pdfkeywords={},
 pdfsubject={},
 pdfcreator={Emacs 31.0.50 (Org mode 9.7.10)}, 
 pdflang={English}}
\begin{document}

\maketitle
\section*{Document Status and Version}
\label{sec:org91e1114}
\begin{quote}
This document is actively maintained. See \textbf{\textbf{Status \& Changes}},  Section
\ref{sec:status-changes}, for recent updates and roadmap.
\end{quote}

\begin{itemize}
\item \textbf{\textbf{Document Date}}: This export was created on \today.
\item \textbf{\textbf{Document Version}}: This export is based on version \docversion{}
of the document source.
\end{itemize}

\thispagestyle{empty}
\newpage
\section*{Table of Contents}
\label{sec:orgc545899}
\setcounter{tocdepth}{2}
\tableofcontents

\newpage
\section{Overview}
\label{sec:org2e326ed}

This document is concerned with a specific risk Midnight faces: the
risk of selecting committees with low fault tolerance, thereby
increasing the risk of a breakdown of the consensus protocol,
\grandpa, that finalizes blocks.  Finality plays a critical role in
Midnight because, unlike most blockchain clients, Midnight's
smart-contract client \texttt{midnight.js} is unable to process rollbacks.
When the client observes a confirmed transaction, it applies pending
local state updates in a destructive manner.  Because of this (perhaps
temporary) limitation, Midnight has been designed to assume fast
finalization to reduce user-visible latencies while protecting against
data loss.  Fast finality also provides direct usability benefits
for people interacting with the chain, so Midnight's commitment to
fast finality is never likely to decrease.

Figure \ref{fig:orgc043d78} shows a simplified set of Midnight components.  The
Midnight Node observes its Cardano Node via DB Sync, and does so at a
delay of 12 hours because that is Cardano's longest-chain finality
window.  But most Midnight clients are more concerned with the
finality delay for transactions they post to the Midnight Node.
Midnight clients rely on the Indexer to provide only finalized
Midnight transactions, to prevent possible data loss should rollback
occur on the Midnight Node.  Anything that disrupts the orderly
finalization of blocks will impact clients.  If \grandpa{} experiences
liveness issues that delay finalization for unbounded periods,
Midnight will appear offline to its clients even if new blocks are
being produced.  If \grandpa{} experiences a safety violation that
allows it to finalize contradictory forks, clients will struggle to
recover.

Although this document mentions \grandpa{}, none of the analysis is
specific to that protocol.  The entire analysis is relevant to any
consensus model that uses quorum-based BFT with stake-weighted,
decentralized committee selection.  This includes the potential
replacement for \grandpa{} and \aura{} being prototyped as part of the
FastBFT effort, based on Jolteon consensus.


The rest of this document presents a detailed analysis of Midnight's
finality risks arising from stake-weighted committee selection.  After
analyzing the inherent risks of that approach, we describe a variant,
Ariadne, developed for Midnight.  We analyze its effectiveness in risk
reduction and offer recommendations for dynamic tuning of Ariadne's
parameters for best effect.  We conclude with an overview of how this
document's analyses fit within a larger risk model of quorum-based
voting in Midnight.


\begin{figure}[htbp]
\centering
\includegraphics[width=0.95\linewidth]{Midnight_Architectural_Components/2025-02-09_14-14-57_screenshot.png}
\caption{\label{fig:orgc043d78}Major components of the Midnight architecture.}
\end{figure}
\section{Finality Risk}
\label{sec:orgddb06af}

To understand finality risk, we must understand how the consensus
protocol makes progress and provides safety.  We will then review the
ways in which the protocol can fail, and the effects of those failures.
\subsection{Quorum-Based Consensus and Weighted Selection}
\label{sec:org61f9f38}

Midnight uses a proof-of-stake consensus where the amount of control
wielded by a community member is proportional to their stake relative
to the stake of other participants.  Unlike Cardano's Ouroboros proof
of stake, which is a longest-chain model, Midnight uses quorum-based
consensus to achieve consensus with faster finality.  Quorum-based
consensus in the presence of adversarial actors is a well-studied
problem, often called \emph{classical BFT}, where \emph{BFT} means \emph{Byzantine
fault tolerance.} A \emph{Byzantine} fault is any fault other than a crash
fault, so it encompasses both malicious behaviors and faults that
cause a well-intentioned participant to participate incorrectly in the
protocol.

A key research result is that quorum-based BFT consensus requires
\(3f+1\) participants to survive \(f\) concurrent Byzantine faults.
Longest-chain protocols like Ouroboros have a weaker assumption of
only \(2f+1\) participants (also known as a 51\% honest majority), but at
the cost of much longer finality times. Quorum-based voting can achieve
finality in minutes or seconds, while longest chain requires hours.
\subsection{Committee Selection}
\label{sec:orgbc7cf30}
Quorum-based consensus is built on the notion of a \emph{committee}---or
series of committees---whose members propose and vote on the value
for each new slot. (In our case, they propose and vote on the next
block in the chain.)  In a decentralized system, this means there must
exist a \emph{committee-selection} algorithm that chooses a committee from
a set of candidate participants.  Midnight follows an approach where
committee selection is based on \emph{weighted random selection} using each
registered participant's stake as their weight.  The rough outline of
the procedure is:


\begin{itemize}
\item Let \(P = \{(p_1, s_1), ..., (p_n, s_n)\}\) be the set of candidate
participants \(p\) and their stakes \(s\) at the time of committee
selection.  Assume \(P\) is a subset of some larger set \(A\) of all stake
pools.
\item Compute \(S = \sum_{i=1}^n\; s_i\) as the total stake.
\item Compute each participant's weight \(\displaystyle(p_i, w_i) =
  \frac{s_i}{S}\quad\forall (p_i, s_i)\in P\).  Let \(W\) be the set of
all such weights.
\item To fill a committee of size \(k\), make \(k\) separate weighted
selections from \(W\), allowing the same participant to be selected
many times. The resulting committee is an ordered list \(K = [p_1,
  ..., p_k]\) which can include duplicate entries.  We may refer to
\(p_i\) as the \emph{owner} of the \(i^{\text{th}}\) committee seat.
\end{itemize}

We assume the overall system selects new committees at some
configurable interval, running this full algorithm each time.
\subsection{Voting Strength}
\label{sec:org86e4596}

Participants selected into a committee do not all wield equivalent
voting power in consensus.  This is an important difference from a
typical centralized deployment, where every node's intentions are
trusted equally and every node's exposure to unintentional faults is
roughly equivalent.  Variable voting strengths are a deliberate outcome
of stake-weighted selection.  They reflect a complex set of assumptions
that can be reduced to ``\emph{more stake implies greater trust}.''

The amount of power a committee member holds, called is \emph{voting
strength}, is the percentage of total seats it owns on the committee.
To compute the voting strength of each participant \(p\) in a committee
\(k\), first count the number of seats \(p\) owns in \(K\); this is
expressed as \(\text{count}(p,K)\) or \(C_K^p\) for short.  The voting
strength \(V_K^p\) of a participant \(p\) in a committee \(K\) with size \(k\)
is computed as \[ V_K^p = \frac{C_K^p}{k} \]


Naturally, a candidate who is not selected into a committee has zero
voting strength on that committee.
\subsection{Implications for Risk}
\label{sec:org8445299}


Voting strengths play a central role in the computational model of
risk this document develops.  For a simple preview: suppose a single
participant \(p\) has voting strength \(V_K^p\;=\;0.34\) on a given
committee \(K\).  If \(p\) were to become faulty, the committee would not
be able to achieve consensus on any finality votes, and the system may
exhibit either safety or liveness issues (or both), as discussed in
section \ref{sec:orgf799702}.

In our fault analysis, we are concerned primarily with the ranked
voting strength of committee members.  For a committee \(K\) with distinct
members \(\{p_1, ..., p_m\}\), we analyze the voting-strengths vector
\[
V_K = [V_K^{p_1}, ..., V_K^{p_m}]\qquad \text{sorted in descending order}
\]
This vector allows us to analyze the amount of voting strength at risk
under different fault assumptions.
\subsection{Subsetting of Candidates}
\label{sec:org596aaa0}

Two critical steps on the path to forming a committee act as filters,
reducing the set of active participants.  Starting with the full set
of potential candidates (say, Cardano SPOs in the case of Midnight),
the first filter is whether or not these potential candidates register to
be considered for committees; we call this self-selected subset the
\emph{participant pool}.  Then weighted selection acts as another filter,
almost certainly subsetting the participant pool due to the low
probability of selecting candidates with very low relative stake for
any committee seats.  Only those candidates who win committee seats
are active participants in consensus.\footnote{The committee members are
also the active participants for block production on Midnight, but the
focus of this paper is their participation in finality voting.}

The subsetting performed by weighted random selection means some
candidate participants will be excluded from some committees.  This is
true even if the committee size \(k\) is much larger than the number \(n\)
of candidates.  Whenever a candidate with large stake wins an
additional seat, that seat is no longer available to other candidates.
The effect when stakes vary widely is to probabilistically exclude the
lowest-staked candidates from committees.

If we equate trust with amount staked, this probabilistic subsetting
effect increases the security of the chain.  But it comes at the cost
of potentially centralizing authority among the largest players.  This
centralization can be reduced by imposing a non-linearity to voting
strength to create diminishing returns past a certain stakepool size.
Midnight has no direct mechanism like this in its launch plan because
Midnight participants are a self-selected subset of Cardano
stakepools, and Cardano already has a \emph{saturation} mechanism to
implement diminishing returns for extremely large stakepools.

Any mechanism that artificially inflates the voting power of
lower-staked candidates opens the door to Sybil attacks in which a
large stakeholder transfers their stake to a large number of smaller
stakepools it controls, resulting in a much higher aggregate voting
strength than their single stakepool would have commanded.  This risk
justifies Midnight's conservative use of strictly linear scaling when
computing voting strength.  While the linear approach is not a defense
against all forms of Sybil attacks, it avoids introducing a direct
incentive among rational actors to split their stake.
\subsection{Quorum Assumptions and Pathologies}
\label{sec:orgf799702}

For the purposes of this discussion, we will write \emph{QBFT} to refer to
the broad class of quorum-based BFT consensus protocols, to
distinguish them from longest-chain BFT protocols such as Ouroboros.
Further, we will refer to \emph{QBFT} as if it were a single protocol, but
our scope includes many practical quorum-based protocols; the
properties we discuss here are common to such protocols.

QBFT consensus assumes greater than \(2/3\) of the committee are \emph{honest}
in the strictly technical sense of \emph{operating within the specification
of the protocol}.  Committee members with the best intentions may fail
to participate honestly if, for example, they experience faults (crash
faults or otherwise), or if their network connectivity suffers due to
benign or malicious causes.

QBFT does nothing to ensure greater than \(2/3\) honest participation or
to cope with situations where \(2/3\) or less of the committee is
behaving honestly.  Those responsibilities fall to the system in which
QBFT is a component.  From the point of view of the QBFT protocol
designer, as soon as the security assumption is violated, ``all bets
are off'' and protocol behavior is wholly unspecified.  In reality, as
we explore below, QBFT systems degrade in some very specific ways
depending on how the security assumption is violated.  Midnight, as a
system leveraging QBFT, must bring its own methods of fostering
greater than \(2/3\) honest participation, and must detect and recover
when this security assumption fails.
\subsubsection{What Can Go Wrong?}
\label{sec:org57a3ca4}

When a QBFT system's honest-participation assumption is violated, its
behavior is unspecified.  The ill effects of this unspecified behavior
are typically classified as either liveness or safety violations, as
described below.
\paragraph{Liveness Risk}
\label{sec:org2167b47}

Liveness risk is the risk that the system will enter a state where
progress is impossible. In a strict technical sense, this usually
refers to a terminal state (or cycle of states) from which there is no
recovery.  In our discussion we are also concerned with transient,
intermittent states in which no finality votes are possible until
either (a) faulty committee members return to normal behavior, or (b)
a new committee is formed with sufficient honest members to restore
progress.

A QBFT that experiences \(f\geq 1/3\) crash faults is below quorum and
can no longer agree to finalize new blocks.  Depending on the details
of the block-production rules, either the system will stop extending
the chain with new blocks or it will simply stop agreeing that any new
blocks are finalized.  In the time it takes the faulty nodes to detect
their faults and return to operation, progress on block finalization
may be delayed arbitrarily.  Also, there is no inherent mechanism to
exclude faulty nodes from each new committee, so repeated committee
selection is likely to keep including faulty members, prolonging the
system liveness fault.

If crash faults alone are problematic, Byzantine faults can be worse.
The operator of a node experiencing an unintentional Byzantine fault
will have a harder time detecting the fault than they would a crash
fault, delaying restoration of normal behavior.  If, instead,
adversarial operators deliberately withhold votes, they may block (or
help block) progress for as long as they are assigned sufficient
voting strength in new committees.  For example, a single committee
member with 34\% voting strength can block all progress by ceasing to
submit votes.

These potentially unbounded interruptions of the normal process of
finalizing blocks are considered liveness faults in our analysis.
\paragraph{Safety Risk}
\label{sec:org26bebac}

A QBFT protocol is \emph{safe} if no two honest participants or observers
can witness the protocol making contradictory decisions.  In the case
of a blockchain, let \(b^\text{\sf F}_i\) be a block at height \(i\) that
any witness can observe as having been finalized by the protocol and
let \(b \prec b'\) signify that \(b\) is in the prefix chain of \(b'\).  The
protocol is \emph{safe} if for all finalized blocks \(b^\text{\sf F}_i\),
\(b^\text{\sf F}_j\), and \(b^\text{\sf F}_k\) where \(i < j < k\), it holds
that \(b^\text{\sf F}_i \prec b^\text{\sf F}_j \prec b^\text{\sf F}_k\).
Simply: the protocol is safe if it does not finalize conflicting
forks.

A QBFT system experiencing \(f \geq 1/3\) Byzantine faults might fail
safety if some of the faulty nodes \emph{equivocate}, which means voting to
support conflicting proposals.  This is the classic form of Byzantine
fault, where observers are somehow partitioned such that they are
unable to compare notes and discover that a duplicitous voter is
voting differently on both sides of the partition.  This could lead
observers in the two partitions to witness quorums and support
conflicting finality decisions.  While \emph{Byzantine} faults refer to any
non-crash faults, equivocation is the key safety risk and is therefore
the primary focus of Byzantine fault tolerance.
\section{Objectives and Measures}
\label{sec:org87286a2}

Before we manage a risk, it is useful to quantify it.  In this section
we identify the most urgent risk objective and our approach to
measuring it.  The same measurement approach should apply to the
system both before and after the introduction of preventive or
mitigating controls.  In later sections we will analyze the inherent
risk (before controls) and residual risk (with controls).
\subsection{Objective: Liveness in the Face of Crash Faults}
\label{sec:orgbf60411}

In a system that could easily face deliberate attacks against liveness
and safety, there is significant value in focusing on a much simpler
path to failure: the risk of crash faults that block new finality
votes.

It may seem counterintuitive to focus on these benign scenarios, but
it helps to consider that no system that can easily fail for
accidental causes is safe from deliberate attacks.  Surviving ordinary
faults is the baseline for all further risk management.

Put differently, a system that requires the simultaneous failure of
\(f\) benign nodes has also made it impossible for an adversary to
realize an attack without controlling more than \(f\) committee seats,
either directly or through corruption.  Anything we do to drive \(f\)
higher to protect against liveness failures among well-intentioned
committees also reduces the chance of a successful attack on the
liveness or safety of the system.  Figure \ref{fig:hierarchy}
illustrates this relationship.


\begin{figure}
  \centering
  \[
  \boxed{\text{accidental liveness}} \gg\ \boxed{\text{deliberate liveness}} \gg\ \boxed{\text{deliberate safety}}
  \]
  \caption{\it A risk hierarchy.  If we cannot defend against accidental liveness faults, we have no defense against deliberate liveness attacks.  If we cannot defend against deliberate liveness attacks, we have no defense against deliberate safety attacks.  That is why we focus first on accidental liveness faults.}
  \label{fig:hierarchy}
\end{figure}
\subsection{Primary Risk Measure: Number of Faults Tolerated}
\label{sec:org9c5a5b0}

Our central risk measure is the probability of selecting a committee
in which crash faults of fewer than \(f\) nodes can result in a liveness
failure.  The target \(f\) is an organizational decision, part of the
organization's \emph{risk appetite}.


Since committees are chosen randomly from participants, which are a random subset of all SPOs, \uline{our aim is to predict the probability of a random committee arising that cannot survive a target minimum number of faults}.  Put differently, we will analyze the highest number of faults that a potential committee can be expected to survive with a target probability.  This lets us pick a risk tolerance of, say 99\%,
and aim for the highest \(f\) that can be achieved with 99\% probability.
\subsubsection{Computing the Number of Faults Tolerated}
\label{sec:orgb1da8f4}


An intuitive way to understand the number of faults a given committee
  \(K = [p_1, ..., p_k]\) can handle is to first derive \(V_K =
  [V_K^{p_1}, ..., V_K^{p_m}]\), which are the voting strengths of each
  committee member, sorted from highest to lowest.  We can then
  compute \(f\) by running the algorithm shown in Listing
  \ref{lst:org12dd7a2}.  The algorithm finds the longest prefix of \(V_K\)
  whose total voting strength is less than \(1/3\) the total voting
  strength of the committee.  The length of that prefix is the number
  of faults the committee can tolerate.  Figure \ref{fig:org5bd7103}
  illustrates this process in two scenarios with different degrees of
  fault tolerance.




\begin{listing}[htbp]
\begin{minted}[]{python}
x, f = 0
for v in V:
  if x + v >= 1/3:
     break
  x += v
  f += 1
return f
\end{minted}
\caption{\label{lst:org12dd7a2}Counting Faults:  This algorithm processes voting-strength vector \(V_K = [V_K^{p_1}, ..., V_K^{p_m}]\) for a committe \(K\), sorted in descending order. The value returned is greatest number of faults that the system is guaranteed to survive.  The committee may survive more faults if they occur in lower-strength committee members, but \(f+1\) faults are capable of violating the security assumptions of the system.}
\end{listing}


\begin{figure}[htbp]
\centering
\includegraphics[width=.9\linewidth]{Objectives_and_Measures/2025-02-10_14-50-11_screenshot.png}
\caption{\label{fig:org5bd7103}Counting Faults: This figure illustates the process for counting tolerated faults.  Every blue bar is a voting strength in \(V_K\), the vertical divider is the final index of the loop, and the value for \(f\) is the count from the prior step.  Read this as ``if all the committee members left of the mark fail, the system has failed.''  As long as only \(f\) faults occur, the system still has sufficient voting power to finalize blocks.}
\end{figure}
\paragraph{Key Concept: What \(f\) Means}
\label{sec:org1f39564}

Looking at Figure \ref{fig:org5bd7103}, one might question why we say
scenario 1 cannot survive a single fault when it could clearly survive
a fault in its weakest committee member.  In fact, if we simply
sorted the members in increasing order, our loop to discover \(f\) would
return much higher values for \(f\).  Why do we sort in descending order
when determining the value of \(f\)?
Think of \(f\) as a number of blank termination notices you give to an
adversary.  The adversary can halt any \(f\) nodes they like.  Your job
is to pick the highest possible \(f\) such that your adversary is unable
to put the system in a faulty state regardless of which members they
halt.  That is the point of \(f\): you always know you can lose \emph{any}
\(f\) nodes and maintain operation.
\section{Inherent Risk}
\label{sec:org44f9df9}


\subsection{Approach}
\label{sec:org630ce7f}

To analyze inherent risk we perform a Monte Carlo simulation of the
process of selecting random committees with different parameters.  The
simulation details are:

\begin{enumerate}
\item Our SPO data is the set of all SPOs on Cardano at the time of a
scrape of \texttt{pooltool.io} at some point in September 2024.
\item We then simulate committee selection assuming different numbers of
registered Midnight participants.  We use a range from fifty up to
two thousand, out of roughly 3000 total SPOs.  Committee size seems
to have fairly little impact on fault tolerance, so we just assume
committee size is 300.
\item For each committee generated, we compute the number of faults the
committee can tolerate.
\end{enumerate}

We then aggregate data from thousands of such simulation runs to build
a probabilistic model indicating the risk that a committee drawn from
a registered participant pool of each size will result in a committee
with low fault tolerance. This provides the baseline data shown in
Figure \ref{fig:org581104c}.  In this chart, 1.0 means 100\% of the simulations
tolerated the given number of faults.
\subsection{Analysis}
\label{sec:org2e376cb}

It's clear from the chart that more participants lead to higher fault
tolerance, even with a fixed committee size of 300.  At the same time,
it's important to note that a centralized consensus protocol could
tolerate 21 faults with 124 nodes (\(3f+1\)), whereas our decentralized,
stake-weighted participation approach requires more than 1000 to reach
the same degree of fault tolerance (closer to \(300f\)).

If Midnight could launch with 1000 active participants, the risk
analysis would focus on making their committees survive even more
faults.  But in the early stages of Midnight it's more realistic to
expect 200 or fewer participants.  We should be looking for
early-stage alternatives or augmentations to simple stake-weighted
committee selection that yield committees whose fault tolerance is
similar to that of a pure stake-weighted model with 1000 or more
participants.




\begin{figure}[htbp]
\centering
\includegraphics[width=.9\linewidth]{Inherent_Risk/2025-02-10_17-28-28_screenshot.png}
\caption{\label{fig:org581104c}Inherent risk that a random committee will not tolerate \(f\) concurrent faults, for \(f\) in 1-50. The different curves represent different numbers of registered participants.  (We acknowledge the labeling here may be confusing.  The ``risk'' is higher at higher values of \(f\), as the chance of tolerating that many faults decreases.  This chart matches the residual risk charts in later sections, but we may update all these charts and wording in a later revision.)}
\end{figure}
\section{Ariadne for Risk Prevention}
\label{sec:org1eefbc0}
The inherent risk analysis shows that adding more participants
increases the fault tolerance of the system.  Ariadne\footnote{Unpublished,
IOG internal.  Ariadne was conceived as a stepping stone to true
multi-resource consensus in Minotaur, designed by Zajkowski and Rossie
circa 2022 for Midnight.  In classical mythology, Ariadne gave Theseus
the ball of string he used to escape the Minotaur's labyrinth.} is a
modification to committee selection that adds a relatively small
number of permissioned nodes holding sufficient stake to stand in for
the eventual SPO participants until they arise.

\begin{figure}[htbp]
\centering
\includegraphics[width=0.5\linewidth]{Ariadne_for_Risk_Prevention/2025-02-10_22-20-36_screenshot.png}
\caption{\label{fig:orgf157bb7}Ariadne's two key parameters are \(\varphi\), the percentage of voting strength reserved for the federated nodes, and \(F\), the number of federated nodes across which Ariadne distributes this voting strength.  If \(S\) is the sum of the stake held by all SPO participants, \(\varphi S\) is the virtual stake held by federated nodes on the committee.  The voting strength of each federated member is \(\frac{\varphi S}{F}\).}
\end{figure}



\begin{figure}[htbp]
\centering
\includegraphics[width=.9\linewidth]{Ariadne_for_Risk_Prevention/2025-02-10_22-17-24_screenshot.png}
\caption{\label{fig:org87af995}Ariadne-inserted committee members dilute the voting strength of the highest-strength members in order to raise the overall fault tolerance of the commitee.}
\end{figure}

Figure \ref{fig:org87af995} shows a committee in which Ariadne has
reserved five committee seats for its auxiliary \emph{federated} members,
and has assigned each an equal amount of \emph{virtual stake}.  The effect
is to dilute the voting strength of the more strengthful permissionless
members, raising the overall fault tolerance.

Figure \ref{fig:orgf157bb7} illustrates the two key parameters that
control Ariadne.  The first is \(F\), the number of committee seats
reserved for federated members.  The second is \(\varphi\) which
represents the percentage of total voting strength to be held across
those federated members.\footnote{The actual configuration parameter that
defines \(\varphi\) is the ``D parameter'', which is specified as a ratio
of permissioned to permissionless seats.}  Ariadne divides the
federated voting strength evenly across the \(F\) federated members.

Ariadne follows these steps to create a committee of size \(k\):
\begin{enumerate}
\item Reserve \(\varphi k\) committee seats for the federated nodes. (These
reserved seats may be randomly distributed across the \(k\) committee
seats.)
\item Use weighted random selection to populate the unreserved
permissionless \((1-\varphi)k\) seats in the usual way, choosing from
the available participants and allowing the same participant to
occupy multiple seats.
\item Populate the reserved federated seats in round robin from the set
of \(F\) federated nodes, again allowing the same participant to
occupy multiple seats.
\end{enumerate}

This will result in a committee where each federated node has a voting
strength of approximately \(\frac{\varphi k}{F}\), regardless of the
voting strengths of the permissionless nodes.  When the voting
strengths are sorted, the federated nodes will sort together in a
block as illustrated in Figure \ref{fig:org87af995}.

The federated nodes must be explicitly configured in Midnight.  They
are sometimes called \emph{trusted} nodes because they are expected to
behave honestly within their ability.  But as with any nodes,
federated nodes are subject to their own faults.  The fault analysis
of an Ariadne committee therefore follows the same approach used in
our study of inherent risk, but with federated nodes inserted in the
correct sorted position within the list of committee members, as
illustrated in Figure \ref{fig:org87af995}.
\section{Ariadne Residual Risk}
\label{sec:org26add81}

To measure the risk reduction offered by Ariadne, we use Monte Carlo
simulations to show the effects of Ariadne with changing parameters.
\subsection{Simulation Parameters and Approach}
\label{sec:orgfaf99c1}

The key variables we test are:
\begin{enumerate}
\item The number of participating SPOs, drawn from the Sept 2024 pooltool scrape.
\item The \(\varphi\) parameter---the percentage of voting strength reserved for
the federated nodes.
\end{enumerate}

We keep two important parameters fixed:
\begin{enumerate}
\item We set the number of federated nodes at 10.  That number is \(3f+1\)
for \(f=3\), allowing the federated nodes to act as their own
fault-tolerant cluster for ancillary consensus problems that can
support Midnight in other ways.  Ten is a reasonable number in
terms of operational cost and complexity.  More is always better
from a risk perspective, but practical concerns suggest a number
closer to 10.
\item We fix the committee size at 300, as it has little effect on the
outcomes (as discovered in prior experimentation).
\end{enumerate}
\subsection{Simulation Results}
\label{sec:org6645fbe}

Here we present our results as a series of charts, each showing the
fault tolerance for a given SPO participation level across a range of
possible \(\varphi\) values.  For each curve, the point at which it drops
below \(1.0\) is the point where uncertainty creeps in as to whether a
committee selected with those parameters will tolerate the number of
faults indicated on the x-axis.


\begin{figure}[htbp]
\centering
\includegraphics[width=0.5\linewidth]{Ariadne_Residual_Risk/aa50.png}
\caption{\label{fig:org17ef00b}Fifty SPOs.  With such low participation, high federated strength is essential to provide any reasonable degree of fault tolerance. Even with 49\% federated, there is a non-zero chance of choosing a commitee with \(f<4\).}
\end{figure}




\begin{figure}[htbp]
\centering
\includegraphics[width=0.5\linewidth]{Ariadne_Residual_Risk/aa100.png}
\caption{\label{fig:orge1ca2a5}One hundred SPOs.  Even with 100 participants, 49\% federated strength still gives the best chance of surviving up to five faults, although 34\% has non-zero chance of surviving more than seven faults, where 49\% cannot.}
\end{figure}




\begin{figure}[htbp]
\centering
\includegraphics[width=0.5\linewidth]{Ariadne_Residual_Risk/aa200.png}
\caption{\label{fig:orgddfa27a}Two hundred SPOs.  Now we see 34\% federated strength as the best way to ensure higher fault tolerance.  It's conclusively better than 49\% at this level of participation.}
\end{figure}




\begin{figure}[htbp]
\centering
\includegraphics[width=0.5\linewidth]{Ariadne_Residual_Risk/aa300.png}
\caption{\label{fig:org566dd54}Three hundred SPOs.  Now 20\% federated strength is a strong contender.  While 34\% still wins for maintaining 100\% tolerance longer, it plummets to zero chance for fault tolerances where 34\% still offers some odds of safety.}
\end{figure}

\begin{figure}[htbp]
\centering
\includegraphics[width=0.5\linewidth]{Ariadne_Residual_Risk/aa500.png}
\caption{\label{fig:org96afb1d}Five hundred SPOs.  Now 20\% federated strength is clearly superiod.  Higher values of \(\varphi\)  have definitely shown their cliff-edge behavior at much lower degrees of fault tolerance.}
\end{figure}




\begin{figure}[htbp]
\centering
\includegraphics[width=0.5\linewidth]{Ariadne_Residual_Risk/aa1000.png}
\caption{\label{fig:org80369bc}One thousand SPOs.  At this level of participation Ariadne's federated voters do more harm than good.  The right strategy is to give them no votes at all.}
\end{figure}




\begin{figure}[htbp]
\centering
\includegraphics[width=0.5\linewidth]{Ariadne_Residual_Risk/aa2000.png}
\caption{\label{fig:org5530ffc}Two thousand SPOs.  (Scale extended up to \(f = 50\) for this one chart.)  In case Midnight achieves this level of participation, the system without federated voters can tolerate up to nearly 30 faults with certainty, with probabilities falling off thereafter.  It reaches zero chance of a safe committee when \(f\) is in the mid 40s.}
\end{figure}
\subsection{Analysis}
\label{sec:org29eb7c5}

These charts tell a consistent story about the benefits of
Ariadne.  When participation is low, there is benefit in a high degree
of federated voting strength. But as the set of SPOs grows, the curves
representing higher \(\varphi\) values all exhibit a vertical cliff at some
point.  As we keep adding SPOs, the curves that employ Ariadne become
less and less effective, with the higher \(\varphi\) curves having the worst
fault tolerance toward the end.

Figure \ref{fig:org08aa174} illustrates why this happens.  As the natural
pool of SPOs grows, the amount of virtual stake held by the federated
committee cannot scale horizontally across more federated nodes, so it
effectively scales vertically.  In the sorted representation of voting
strengths, this shifts the whole block of federated committee members
farther and farther to the left.  When they become the leading voters,
they also become the weakest link in the fault tolerance story.
\subsection{Recommendations}
\label{sec:org25e4e93}

These simulations show that Ariadne is not a set-and-forget system.
It must be continuously monitored and updated to ensure it is a net
benefit to the fault tolerance of the system, or else it can become a
significant detriment.  The best way to make the right changes is to
track actual participating SPOs and potential non-participating ones,
and to run simulations like those used here to ensure the current
\(\varphi\) is still advantageous.  In the early stages of the network
we might have a \(\varphi\) of 34\% or even 49\%, but we may need to drop
to 20\% shortly thereafter and we may even go to 0\% if we are lucky
enough to have thousands of participating SPOs.



\begin{figure}[htbp]
\centering
\includegraphics[width=0.95\linewidth]{Analysis/2025-02-10_23-25-52_screenshot.png}
\caption{\label{fig:org08aa174}Rough illustration of the problem of too much federated voting strength when there is high SPO participation.  When \(\varphi\) is too large and \(F\) is too small, the federated nodes sort to the front of the list and dominate the fault tolerance of the system.  Removing them re-normalizes voting strength to the permissionless members, creating higher fault tolerance at higher SPO participation levels.}
\end{figure}
\section{Toward a Full Model of Finality Risk}
\label{sec:org8b11170}

A complete model of Midnight's finality risk is outside the scope of
this document, but the elements discussed here form one of two pillars
for such a model, each of which is largely independent of the other.
The rest of this section provides a brief overview of these pillars
and how their results can be combined to form a full risk model.
\subsection{The Risk of a Committee that Cannot Tolerate \(f\) Faults}
\label{sec:org816e71c}

This risk is the focus of the current document.  The document provides
an approach to analyzing the risk of selecting committees with low
fault tolerance, and establishes both the inherent risk and the risk
reduction that can be achieved using Ariadne.
\subsection{The Risk of \(f\) Faults Occurring at Once}
\label{sec:org303d75f}

This risk is outside the scope of the current document.  It's
essentially open-ended problem with a great deal of nuance.  We cannot
do justice to the topic here, but it might be useful to consider some
of the factors to consider in such a model.  For most of this
discussion we are concerned with improving the \emph{honest uptime} of
nodes operated by honest actors; it will be clear when we mean to
discuss adversarial actors.
\subsubsection{MTBF Abstraction}
\label{sec:org207453a}

For a node to become a faulty protocol participant requires either
adversarial intent of the failure of some component or interaction
among components within the node or the network.  Nodes are complex,
consisting of many interacting components with their own fault
profiles.  A deep analysis of all these potential faults can be
instructive, but can never be fully complete.  Instead, it's common to
abstract over root causes by simply measuring the mean time between
failure (MTBF) of entire nodes and of their major components. This is
much more tractable than trying to analyze the complex interactions of
all hardware, software, and network elements.
\subsubsection{Detection Time}
\label{sec:org711cdce}

Byzantine faults are, by nature, hard to detect.  It's hard for other
nodes to reliably detect a Byzantine fault in another node, but it's
also hard for an operator to detect a Byzantine fault in their own
node.  Since nodes and components suffering Byzantine faults continue
to attempt protocol interactions, but do so incorrectly, fault
detection requires someone to monitor KPIs related to successful
protocol interactions.  Even when those KPIs indicate an issue, the
node operator must determine whether they are at fault or whether they
are witnessing faults in other nodes.  The longer it takes a node
operator to detect their own faults, the longer they will remain in a
faulty state.  The average time to detect a fault is called the mean
time to detection (MTTD).
\subsubsection{Recovery Time}
\label{sec:org21f5739}

The mean time to recovery (MTTR) is essential in a model meant to
predict the risk of concurrent faults.  The sum of MTTD and MTTR is
the total time the node is not experiencing honest uptime.  

If MTBF is much shorter than MTTD+MTTR, it's easy to imagine a system
in which all nodes are in recovery at the same time.  If MTBF is much
greater, it helps reduce the probability of concurrent faults.
\subsubsection{DB Sync Example}
\label{sec:orgc1e34cf}

Every Midnight node has, for example, a DB Sync instance it relies on
for information about Cardano.  While it may be possible to replace DB
Sync with a different technology, current nodes can fail simply
because DB Sync fails.  DB Sync instances are known to fail silently,
producing incorrect data rather than crashing.  By definition, these
are Byzantine faults.  They manifest as incorrect participation in the
protocol, causing the node's votes and block-production opportunities
to be wasted, and preventing the node from validating correct traffic
from the rest of the network.

The honest uptime of a node can be improved by a number of approaches,
including:
\begin{enumerate}
\item Raising DB Sync's MTBF (through hardening engineering on that codebase).
\item Reducing the node's MTTD for DB Sync.
\item Reducing the MTTR to get the node back into service.
\item Replacing DB Sync with a component whose MTBF is higher and/or
whose MTTD+MTTR is lower.
\end{enumerate}

One possible mechanism, as an illustration, would be for a set of
trusted nodes to operate a small fault-tolerant cluster that
periodically votes on the current hash of the important elements of DB
Sync's state and publishes a signed outcome of their agreed value to
Midnight's gossip network.  This would allow Midnight nodes to
automatically self-check against a known-good finterprint to greatly
reduce MTTD.  Such a cluster could also publish full state snapshots
of their DB Sync instances (using, say \texttt{psql\_dump}) to provide fast
recovery, reducing MTTR.
\subsubsection{Common Cause Analysis}
\label{sec:org0912456}

Independent node failures are less of a worry than failures from
common causes.  If multiple nodes share any common infrastructure or
use common operational support, a failure in that infrastructure or
support can affect all those nodes concurrently.  In a decentralized
system these common-cause faults are extremely hard to predict,
because they are the results of individual choices by decentralized
actors.
\subsubsection{Adversarial Control or Collusion}
\label{sec:org2dc434a}

These are important variants of common-cause faults.  In the first
case, an adversarial actor corrupts and controls the votes of other
participants.  In the second case, a number of adversarial actors
actively collude to combine their voting power.  In either case, such
adversarial behaviors fall outside MTBF/MTTR analysis and form a
category of their own.  The assumption that ``\emph{more stake implies
greater trust}'' must somehow factor into any probability analysis for
collusion, but says little about adversarial corruption and control.
These are perhaps the hardest risks to model accurately in a
decentralized chain.
\subsection{Putting it Together}
\label{sec:org7720606}

Suppose we can quantify these risks as probabilities, where
\(P(\text{not tolerated})_f\) is the probability of selecting a
committee cannot tolerate \(f\) concurrent faults, and
\(P(\text{occur})_f\) is the probability that \(f\) faults might occur
simultaneously.  If these turn out to be independent probabilities, we
can compute the probability of both arising as the product of their
probabilities:

\[
P(\text{system fault})_{f} =  P(\text{not tolerated})_f \times P(\text{occur})_{f}
\]

There remain some subtleties to explore before we could accept this
simple model.  Suppose, for example, larger stakepools have lower risk
of individual faults.  Then the overall model becomes more complicated
because the system can generally support more faults among smaller
stakepools.
\subsection{Telling Risks Apart}
\label{sec:org5ce8efc}

It's not always easy to see whether a risk is in one pillar or
another, although they are often easily distinguished.  Consider a
scenario in which the single operator of several large stakepools
participates in Midnight for a period of time and then simply stops
without de-registering.  There is a real risk the committee-selection
protocol will continue to add them to committees despite their lack of
activity, since selection does not consider such factors.  The result
could be a persistent state in which all their committee members
represent perpetual, concurrent faults that the system must carry.

This risk is harder to address than it may seem; it requires a
protocol for proposing and challenging claims that a committee member
appears inactive, as well as a protocol to allow them to return when
they are ready.

We might say that this scenario concerns the risk of choosing a
committee that cannot tolerate \(f\) faults, since it concerns a
blind spot in the committee selection algorithm.  But in fact it is
primarily about the risk of \(f\) concurrent faults because the blind
spot has the potential to allow faults to accumulate.  The underlying
issue is that the candidate has abandoned the protocol, which is a
fault.  
\subsection{Conclusion}
\label{sec:org8d5764b}

This was only a brief overview of the larger risk analysis problem,
but it should serve to contextualize the contributions of the current
document.  Regardless of the specifics of such a combined model, it
should be clear that raising the number of faults a committee can
tolerate is a significant part of risk management.

\newpage
\appendix
\section{Status \& Changes}
\label{sec:orgcc11962}
\label{sec:status-changes}
\subsection{Version History}
\label{sec:org2c2fff8}
\begin{itemize}
\item \textbf{\textbf{\docversion{} (\today)}}: Significant update after early feedback.
\begin{itemize}
\item New title more reflective of the focus of the document within a larger risk model.
\item New section ``Toward a Full Model of Finality Risk'' that
contextualizes this document's contribution within a larger risk
analysis.
\item Finished incomplete paragraph concerning participant subsetting.
\end{itemize}
\item \textbf{\textbf{1.01 (2025-02-14)}}: Updated page 16 with the correct
procedure for choosing a committee in Ariadne.
\item \textbf{\textbf{1.0 (2025-02-14)}}: Initial version.
\end{itemize}
\subsection{Future Roadmap}
\label{sec:orgac4773f}
\begin{itemize}
\item \textbf{\textbf{Revisit Risk Charts}}: The charts for inherent and residual risk
might be inverted.  That is, if we say a chart is about risk, then
100\% should be total exposure, while 0\% should be no exposure.  We
should revisit the graphs and wording to be more clear.
\item \textbf{\textbf{Incorporate Unused Charts}}: Find the right place to include the
unused charts in section \ref{sec:orga658258}.
\item \textbf{\textbf{Simulation Details}}: Improve description of the residual risk
simulation setup.
\item \textbf{\textbf{Simulation Details}}: Improve description of the residual risk
simulation setup.
\item \textbf{\textbf{Improved Analysis and Recommendations for Ariadne}}: That section
is a bit light at the moment.
\item \textbf{\textbf{Mitigation}}: We have discussed Ariadne as a \emph{preventive} risk,
but when something really does go wrong, we need some mitigations
and recovery strategies.
\item \textbf{\textbf{Health Checks}}: Another preventive measure is some basic form of
health check we can perform against committee members before adding
them to the committee.  There are some ideas in the works that need
to be added here.
\item \textbf{\textbf{Equivocation Penalties}}: \grandpa{} allows members to detect and
respond to equivocation; we should discuss here.
\item \textbf{\textbf{Fast Failure Detection}}: When well-meaning nodes fail, they may
fail in a non-crash manner.  We could improve the fault tolerance of
the system by reducing the average time required for an operator to
learn their node has failed.
\item \textbf{\textbf{Fast Failure Recovery}}: Similar to fast failure detection, a node
that returns more quickly to correct operation is one less failed
node for the system to tolerate concurrently.
\end{itemize}
\section{Risk Management Framework}
\label{sec:orgfc8db9e}

This appendix provides a short overview of risk management terminology
and procedures.
\subsection{Risk Decision-Makers \& Appetite}
\label{sec:orged7a27b}
Risk decision-making requires a defined governance structure. Decision-makers establish risk policies, set risk appetite, and enforce risk tolerances. Risk appetite represents the level of risk an organization is willing to accept in pursuit of objectives. Tolerances define acceptable deviations from this threshold.

Risk is not inherently negative. Some risks are necessary to achieve strategic goals. Effective risk management ensures that risks are identified, assessed, and controlled within predefined thresholds.
\subsection{Impact vs. Probability (Severity vs. Frequency)}
\label{sec:org91c406b}
Risk is assessed along two dimensions:
\begin{itemize}
\item \textbf{Impact (severity):} The consequence of a risk event materializing.
\item \textbf{Probability (frequency):} The likelihood of the event occurring.
\end{itemize}

Both dimensions are critical for prioritization. High-impact, low-probability risks may require different strategies than low-impact, high-probability risks. Risk assessments must quantify these attributes where possible to facilitate structured decision-making.
\subsection{Risk Event Types \& Categories}
\label{sec:orgc7754ff}
Risks are classified into structured categories to improve identification, assessment, and response. Classification typically follows a hierarchical structure:
\begin{itemize}
\item \textbf{High-level categories:} Broad risk domains (e.g., operational, financial, security, compliance).
\item \textbf{Subcategories:} Specific risk types within each domain (e.g., data breach under security, liquidity risk under financial).
\end{itemize}

Classification schemes must remain adaptable. New risks may necessitate reclassification, and obsolete categories must be removed to maintain relevance.
\subsection{Inherent Risk (Pre-Control Assessment)}
\label{sec:org4d8445a}
Inherent risk is the level of risk in the absence of mitigating controls. It is evaluated along the impact and probability dimensions. Organizations assess inherent risk to:
\begin{itemize}
\item Understand baseline exposure.
\item Prioritize risk mitigation efforts.
\item Establish a benchmark for measuring control effectiveness.
\end{itemize}
\subsection{Risk Controls \& Residual Risk}
\label{sec:orgd3bc85d}
\textbf{Risk controls} are measures that reduce either the impact or probability of a risk event. Controls include technical, procedural, and administrative mechanisms.

Residual risk is the risk that remains after controls are applied. Effective risk management seeks to ensure that residual risk falls within acceptable tolerances.

Control effectiveness is periodically evaluated to verify that controls function as intended. Weak or obsolete controls necessitate adjustment or replacement.
\subsection{Monitoring \& Adaptive Risk Management}
\label{sec:org265dd9a}
Ongoing monitoring is necessary to maintain an effective risk management framework. This includes:

\begin{itemize}
\item \textbf{Event tracking:} Risk events are recorded and classified according to the established taxonomy.
\item \textbf{Adaptive learning:} Risk categories evolve based on observed incidents and emerging threats.
\item \textbf{Scenario analysis:} Risk exposure is periodically reassessed under varying conditions to identify potential gaps.
\item \textbf{Control assurance:} Control implementation and effectiveness are routinely evaluated to ensure alignment with risk tolerances.
\end{itemize}

Monitoring mechanisms ensure that risk management remains an iterative process, continuously adapting to internal and external changes.
\section{Unused Charts}
\label{sec:orga658258}

This section includes some charts and accompanying discussion that are
not yet integrated into the main document.


\begin{figure}[htbp]
\centering
\includegraphics[width=0.75\linewidth]{Unused_Graphics/2025-02-13_10-57-57_screenshot.png}
\caption{\label{fig:orga8805d4}Scaling effect of federation.  This chart shows the scaled voting strength (red) of all committee members after the addition of five federated nodes with 20\% of the voting strength.  Let \(k=300\) be the committee size.  The blue line represents \(\text{count}(p)/0.2k\) for every unique participant \(p\) in the permissioness committee, with a gap where the federated nodes will ultimately appear.  The red line shows that all nodes are scaled down, but the scaling effect is larger for larger nodes.}
\end{figure}



\begin{figure}[htbp]
\centering
\includegraphics[width=0.75\linewidth]{Unused_Graphics/2025-02-13_11-30-09_screenshot.png}
\caption{\label{fig:org9ec3147}Similar to Figure \ref{fig:orga8805d4}, but with excessive federated strength of 30\%, causing the federated nodes to each have voting strength higher than any SPO.  The specific threshold where a given federated voting strength, such as 30\%, becomes excessive depends on the stake distribution of participating SPOs.}
\end{figure}


The specific threshold where a given federated voting strength, such as \(\varphi =\) 30\%, becomes excessive depends on the stake distribution of participating SPOs.  It's conceivable, but not in current plans, to automate the selection of a computed \(\varphi\) that brings maximal increase in fault tolerance.
\end{document}
